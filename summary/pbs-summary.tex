%%%\documentclass[paper,notoc,12pt]{JHEP3}
%%%mmaetz: put a4paper
%\documentclass[a5paper]{book}
\documentclass[10pt,a5paper,article,twoside]{memoir}
%%%mmaetz: reduce margins
%%%mmaetz: omitted temporarily to have the margins for the change remarks.
%\usepackage[a5paper,left=2cm, right=1cm,top=1.5cm, bottom=1.5cm]{geometry}
%mmaetz: use tikz package
\usepackage{tikz,tikz-3dplot,pgflibraryshapes}
%\usepackage[headings,cm]{fullpage}
\usetikzlibrary{positioning,calc,matrix,chains,scopes,fit,decorations,decorations.pathmorphing,decorations.pathreplacing,arrows,patterns,3d}
%%%\usepackage{pgfplots}
\usepackage[utf8]{inputenc}
\usepackage[ngerman,english]{babel}

%\usepackage{epsfig,cite,amsmath,amssymb,amsbsy,mathrsfs}
\usepackage{epsfig,cite,amsmath,amssymb,amsbsy,mathrsfs}
\usepackage{graphicx}
\usepackage{color}
\usepackage{latexsym}
%%% mmaetz packages
%%% Side captions
\usepackage{sidecap}
%%% Theorems like Theorem 3.1, proof environment
%\usepackage{amsthm}
%% NTheorem is a reimplementation of the AMS Theorem package. This will allow
%% us to typeset theorems like examples, proofs and similar.
%% NOTE: Must be loaded AFTER amsmath, or the \qed placement will break
\usepackage[amsthm,thmmarks]{ntheorem}
%%% Using this packages numbers with units are always typed correctly.
\usepackage{siunitx}
%%% To cross out some stuff.
\usepackage{cancel}
%%% To have an enumerate environment with options to get like.
%%% i) bla
%%% ii) lala
%%% typing just \item
\usepackage{enumerate}
%%% To use \bm\mathrm instead of \vec one needs the appropriate greek letters
\usepackage{upgreek}
%%% bold math
\usepackage{bm}
%%% Nice tabular
\usepackage{booktabs}
%%% For iddots (Inverse diagonal triple dot.)
\usepackage{mathdots}
%%% To change the headers. 
\newcommand{\changefont}{%
	\fontsize{9}{11}\selectfont
}
%%% fancy headers
%\usepackage{fancyhdr}
%\pagestyle{fancy}
%\fancyhead[LE]{\changefont\slshape\nouppercase \rightmark} %section
%\fancyhead[RE]{\thepage}
%\fancyhead[RO]{\changefont\slshape\nouppercase \leftmark} % chapter
%\fancyhead[LO]{\thepage}
%\fancyfoot[C]{}

%%% To use mathclap.
\usepackage{mathtools}
%%% To have a set of bigger braces.
%%% I found out the guy who made that was at at the university of Lille, the city I was born! =)
\usepackage{yhmath}
%%% subfigures, subfloat used in week8.
\usepackage{subfig}

%%% mmaetz: This is for the change notes
%\usepackage[deletedmarkup=none]{changes}

%%% Use this option instead of the above one to make the red stuff disappear and the blue stuff become black. (And also remove the footnotes done with the change package.)
%%% Below are the options used for the change package. Note that the default is to cross out the deleted stuff but this doesn't work well in a math environment so the default settings have been changed.
%%% They have been commented out because all change markups have been removed.
%\usepackage[deletedmarkup=none]{changes}
%\setdeletedmarkup{\textcolor{red!75!black}{#1}}
%\setauthormarkupposition{left}
%\setremarkmarkup{\footnote{#1: \textcolor{Changes@Color#1}{#2}}}
%\setremarkmarkup{\marginpar{#1:#2}}
%%% mmaetz: To suppress the change notes put the final option like that:
%\usepackage[final]{changes}
%%% mmaetz: I'm using an authors id
%\definechangesauthor[name={Marc Maetz},color=blue!50!black]{MM}

%%% mmaetz: Can be useful for something but forgot what and I'm not using it here.
%\usepackage{etoolbox}
%%% mmaetz: Just discovered breqn which provides automatic line breaking. Very nice, more powerful but requires a bit of getting used to it.
%%% flexisym is needed by breqn
\usepackage{flexisym}
\usepackage{breqn}
%%% To make the references clickable
%%% The color settings are in the tikzstuff.tex file.
\usepackage{hyperref}

%\hypersetup{colorlinks=false,linkbordercolor=gray}

%%% mmaetz: For a better readability, I made the vectors bold and roman
%%% note that this is bad practice. Consider something like:
%%% If you find a way to make a curly E upright, you might win the nobel price.
\newcommand{\vtr}[1]{\bm{\mathrm#1}}
\renewcommand{\vec}[1]{\bm{\mathrm#1}}
\newcommand{\s}{\ensuremath{\;}}
\newcommand{\lag}{\ensuremath{\cal L}}
\newcommand{\dslash}{\ensuremath{\displaystyle{\not}}}
\newcommand{\Ta}{\ensuremath{\tilde T}}
\newcommand{\tr}{\ensuremath{\text{tr}}}

%%% These commands have been modified because breqn needs to know if a "|" is left or right for the automatic alignment
\newcommand{\ket}[1]{\ensuremath{\left\lvert {#1} \right\rangle}}
\newcommand{\bra}[1]{\ensuremath{\left\langle {#1} \right\rvert}}
%%% mmaetz: Added negative thin space, added vphantom (makes both sides of the braket of the same height in all cases)
\newcommand{\braket}[2]{\ensuremath{\left\langle {#1}\vphantom{#2}   \middle| {#2}\vphantom{#1} \right\rangle}}
\newcommand{\ketbra}[2]{\ensuremath{ {\ket{#1} \!\bra{#2}}}}
\newcommand{\proj}[1]{\ensuremath{ {\ket{#1} \!\bra{#1}}}}
\newcommand{\sumproj}[1]{\ensuremath{ {\sum_{#1}\proj{#1}}}}
\newcommand{\sand}[3]{\ensuremath{\left\langle {#1}\vphantom{#3} \right\rvert{#2} \left\lvert{#3}\vphantom{#1}\right\rangle}}


\newcommand{\vevj}[2]{\left\langle {#1} \right\langle_{#2} }
\newcommand{\vev}[1]{\left\langle {#1} \right\rangle }


\newcommand{\comm}[2]{\left[ {#1}, {#2} \right] }
\newcommand{\acomm}[2]{\left\{ {#1}, {#2} \right\} }

\newcommand{\norm}[1]{\left\lvert {#1} \right\rvert }

%%% mmaetz: For the differential.
\newcommand{\td}{\mathrm d }
%%% vphantom command to get bigger braces where needed
\newcommand{\vaa}{\vphantom{A^{A}}}
\newcommand{\vaaa}{\vphantom{{A^{A}}^{a}}}
\newcommand{\dls}{\delimitershortfall=-1pt}

%%% calculus: (Part of what I found some time ago somewhere in the internet.)
\newcommand{\ddd}[2]{\frac{\mathrm d ^{2} #1}{\mathrm d #2 ^{2}}}
\newcommand{\dd}[2]{\frac{\mathrm d #1}{\mathrm d #2}}
\newcommand{\ddn}[3]{\frac{\mathrm d ^{#1} #2}{\mathrm d #3 ^{#1}}}
\newcommand{\pdd}[2]{\frac{\partial #1}{\partial #2}}
\newcommand{\pddd}[2]{\frac{\partial^{2} #1}{\partial #2 ^{2}}}
\newcommand{\pdddm}[3]{\frac{\partial^{2} #1}{\partial #2 \partial #3}}
\newcommand{\pddn}[3]{\frac{\partial^{#1} #2}{\partial #3 ^{#1}}}



\newtheorem{expl}{Example}[chapter]
\newtheorem{remk}{Remark}[chapter]
\newtheorem{corl}{Corollary}[chapter]
\newtheorem{thrm}{Theorem}[chapter]
\newtheorem{exer}{Exercise}[chapter]

%% Proof environment with a small square as a "qed" symbol
%\theoremstyle{nonumberplain}
%\newtheorem{proof}{Proof}
%\qedsymbol{\qed}
%\theoremsymbol{\qed}

%% Memoir layout setup

%% NOTE: You are strongly advised not to change any of them unless you
%% know what you are doing.  These settings strongly interact in the
%% final look of the document.

% Dependencies
\usepackage{array}
\counterwithout{chapter}{section}

% Define the default sans serif font as the lighter computer modern bright by
% D. Knuth.
\renewcommand{\sfdefault}{cmbr}

%%% Define a nice orange color and use it for hyperref
%%% Needs xcolor
\usepackage{xcolor}
\definecolor{rioday}{RGB}{255,166,0} 
\hypersetup{colorlinks=false,linkbordercolor=rioday}


% Turn extra space before chapter headings off.

% Chapter style redefinition
\makeatletter
\newcommand\thickhrulefill{\leavevmode \leaders%
\hrule height 6.25pt depth -3.25pt \hfill \kern \z@}
\setlength\midchapskip{10pt}
\makechapterstyle{VZ14}{
  \renewcommand\chapternamenum{}
  \renewcommand\printchaptername{}
  \renewcommand\chapnamefont{\Large\scshape}
  \renewcommand\printchapternum{%
    \chapnamefont\null\thickhrulefill\quad
    \@chapapp\space\thechapter\quad\thickhrulefill}
  \renewcommand\printchapternonum{%
    \par\thickhrulefill\par\vskip\midchapskip
    \hrule\vskip\midchapskip
  }
  \renewcommand\chaptitlefont{\Huge\scshape\centering}
  \renewcommand\afterchapternum{%
    \par\nobreak\vskip\midchapskip\hrule\vskip\midchapskip}
  \renewcommand\afterchaptertitle{%
    \par\vskip\midchapskip
\hrule\nobreak\vskip\afterchapskip}
}

% Set the way pages are layed out (headers and page numbering)
\pagestyle{ruled}
%\if@twoside
  %\pagestyle{Ruled}
%\else
  %\pagestyle{ruled}
%\fi

% Use the newly defined style
\chapterstyle{VZ14}

% Redefine sectional headings to contain rules
%\renewcommand{\section}{\@startsection{section}{1}{0mm}%
%{-2\baselineskip}{0.8\baselineskip}%
%{\hrule depth 0.2pt width\textwidth\hrule depth1.5pt%
%width0.25\textwidth\vspace*{1.2em}\Large\bfseries\sffamily}}

%\renewcommand{\subsection}{\@startsection{subsection}{2}{0mm}%
%{-2\baselineskip}{0.8\baselineskip}%
%{\hrule depth 0.2pt width\textwidth\hrule depth1pt width0.25\textwidth\vspace*{0.8em}\large\bfseries\sffamily}}

%\renewcommand{\subsubsection}{\@startsection{subsubsection}{3}{0mm}%
%{-2\baselineskip}{0.8\baselineskip}%
%{\large\bfseries\sffamily}}

\setparaheadstyle{\normalsize\bfseries\sffamily}
\setsubparaheadstyle{\normalsize\bfseries\sffamily}

% Set captions to a more separated style for clearness
\captionnamefont{\sffamily\bfseries\footnotesize}
\captiontitlefont{\sffamily\footnotesize}
\setlength{\intextsep}{16pt}
\setlength{\belowcaptionskip}{1pt}

%%% Make a bit of additional space for footnotes
\addtolength{\skip\footins}{4pt}
\renewcommand{\footnoterule}{%
   \kern -7pt                   % call this kerna
   \hrule height 0.4pt width 0.4\columnwidth
   \kern 6.6pt                  % call this kernb
}

% Set section and TOC numbering depth to subsection
\setsecnumdepth{subsection}
\settocdepth{subsection}

% Turn off american style paragraph indentation and add some space to be
% printed when a new paragraph starts.

\setlength{\parindent}{0pt}
\addtolength{\parskip}{2pt}

\newcommand{\professor}[1]{\def\@professor{#1}}
\renewcommand{\maketitlehookb}%
{\vspace{2em}\centering\Large\@professor\vspace{0.3\textheight}}

%% This provides a frontend to set the lecture date into the header
%% The chapter names are usually shorter than the section names. So the date should be at this place.
%\newcommand{\lecturedate}[1]{\def\@lecdate{#1}}
%\makeevenhead{ruled}{\normalfont\leftmark,}{}{\@lecdate}
%%% Make the header the same width as the text
%\makerunningwidth{ruled}{\textwidth}
%\makeheadrule{ruled}{\textwidth}{\normalrulethickness}
\renewcommand{\footruleskip}{-5pt}
\makeatother

% This defines how theorems should look. Best leave as is.
\theoremstyle{plain}
\theoremseparator{:\quad}
\theoremprework{}
\theoremindent2em
\theoremheaderfont{\sffamily\bfseries}
\theorembodyfont{\normalfont}
%\theoremsymbol{}
%% Minimal margin to print two pages on an A4 paper or viewing it on tablets.
\settypeblocksize{17.7cm}{11.8cm}{*}
\setlrmargins{2cm}{*}{*}
\setulmargins{1.6cm}{*}{*}
\setheadfoot{7pt}{20pt}
\setlength{\beforechapskip}{-1.2cm}
\checkandfixthelayout

%%% New definition of square root:
%%% it renames \sqrt as \oldsqrt
%\let\oldsqrt\sqrt
%%% it defines the new \sqrt in terms of the old one
%\def\sqrt{\mathpalette\DHLhksqrt}
%\def\DHLhksqrt#1#2{%
%	\setbox0=\hbox{$#1\oldsqrt{#2\,}$}\dimen0=\ht0
%	\advance\dimen0-0.2\ht0
%	\setbox2=\hbox{\vrule height\ht0 depth -\dimen0}%
%	{\box0\lower0.4pt\box2}}

%%% Theorems. (Copied from www.mitschriften.ethz.ch template)


\title{\Huge Physically-based Simulation in Computer Graphics\\\vspace{1em}
Summary of slides
}


\preauthor{}
\author{\LARGE Marc Maetz}
\postauthor{\Large\\\vspace{2em}
\begin{tabular}{ccc}
	\begin{tabular}{c}
		Berhard Thomaszewski,\\
		Amit Bermano\\
		Disney Research Zurich
	\end{tabular}
	&\quad 
	&
	\begin{tabular}{c}
		Barbara Solenthaler\\
		CGL ETH Zürich
	\end{tabular}
\end{tabular}
  %ETH Zurich,\\
  %8093 Zurich, Switzerland\\
	\vspace{1em}
  %E-mail: {\em babis@phys.ethz.ch}
}
\professor{}
%\professor{Prof. Dr. Markus Gross, Prof. Dr. Marc Pollefeys}

\date{\vspace{1em}\today}

%\author{Babis Anastasiou\\
%  Institute for Theoretical Physics, \\ 
%  ETH Zurich,\\
%  8093 Zurich, Switzerland\\
%  E-mail: {\em babis@phys.ethz.ch}}

\begin{document} 

\begin{titlingpage}
\maketitle
	%\titleS
%\vfill
%Revision \SVNRev{} --- \SVNDate{}
\end{titlingpage}


%%% Color settings and plot settings. If the pictures look bad look at this file!
%\definecolor{scolor}{RGB}{255,220,181} 
%\definecolor{scolor}{RGB}{255,166,0} 
%\colorlet{scolordark}{scolor!50!black}

\definecolor{rioday}{RGB}{255,166,0} 
\colorlet{riomorning}{rioday!65!black}
\colorlet{rionight}{rioday!50!black}
%\definecolor{scolor2}{RGB}{0,255,166} 
%\definecolor{scolor2}{RGB}{100,100,170} 
%\definecolor{scolor2}{RGB}{0,233,255} 
%\definecolor{scolor2}{RGB}{0,182,255} 
%\definecolor{scolor2}{RGB}{0,255,16} 
%\definecolor{scolor2}{RGB}{0,89,255} 
%\colorlet{scolor2dark}{scolor2!50!black}

\definecolor{helsinkiday}{RGB}{0,89,255} 
%%% Hi from 2013. I think it's fascinating that this code could possibly be used until like 2043 or even further.
\colorlet{helsinkimorning}{helsinkiday!65!black}
\colorlet{helsinkinight}{helsinkiday!50!black}
\hypersetup{colorlinks=false,linkbordercolor=rioday}
%%%This is the global setting of the number of samples used to plot the functions in the graphs. 
\tikzset{
	smasa/.style={
		%%%The number of samples with low number.
		%%% 1000 should be enough for a final edition
		%samples=10
		samples=1000
	},
	bigsa/.style={
		%%%The number of samples with high number.
		%%% 10000 should be enough for a final edition.
		%samples=20
		samples=10000
	},
	%%%The standard filling
	sfill/.style={
		fill=rioday
	},
	zyplane/.style={canvas is zy plane at x=#1,very thin},
	zxplane/.style={canvas is zx plane at y=#1,very thin},
	yxplane/.style={canvas is yx plane at z=#1,very thin}
}




%\begin{abstract}
%The subject of the course an  introduction to  quantum 
%field theory. The following topics are discussed: 
%\begin{itemize}
%\item Theory of classical fields.  
%\item Canonical quantization of free fields.
%\item The  Dirac equation and  quantization of the Dirac field
%\item Field Propagation, interacting fields and perturbation theory. 
%\item Cross-sections  and decay rates.
%\item Introduction to QED and the problem of infinities.
%\item One-loop renormalization of QED. 
%\end{itemize}
%\end{abstract}

%\tableofcontents 
%\listofchanges
%\bibliographystyle{JHEP}
%\begin{thebibliography}{10}
%\bibitem{srednicki}
%Modern Quantum Mechanics, Sakurai 
%\bibitem{sterman}
%The Feynman Lectures in Physics, Feynman 
%\bibitem{weinberg}
%The Quantum Theory of Fields, Weinberg 
%\end{thebibliography} 
%\newpage

%%% Have decided to reverse the order of the material wrt the first time I taught this. 
%%% Files renamed according to the new order.

%%%%%%%%%%%%%%%%%%%%%%%%%%%%%%%%%%%%%%%%%%%%%%%%%
%%%%% bibliography
%%%%%%%%%%%%%%%%%%%%%%%%%%%%%%%%%%%%%%%%%%%%%%%%%%

\chapter{Finite elements}
\section{Poisson's equation}
\begin{dgroup}[]
	\begin{dmath*}[]
		-\Delta u(x,y)=f(x,y)
	\end{dmath*}
\end{dgroup}
\section{Finite Difference}
\begin{expl}[1D problem]
	\begin{dmath}[]
		-u''(x)=f(x)\condition*{x\in \Omega=(0,1)}
	\end{dmath}
	\begin{dmath}[compact]
		u(0)=u(1)=0
	\end{dmath}
	\begin{dsuspend}
		regular grid
	\end{dsuspend}
	\begin{dmath}[]
		u[i]=u(i\cdot h)\condition*{i\in (0,\ldots,n)}
	\end{dmath}
	\begin{dsuspend}
		approximate derivatives
	\end{dsuspend}
	\begin{dmath}[]
		u''(x)=\frac{u'[i]-u'[i-1]}{h}=\frac{u[i+1]-2u[i]+u[i-1]}{h^2}
	\end{dmath}
	\begin{dsuspend}
		equation for eevry grid point $2,\cdots,n-1$
	\end{dsuspend}
	\begin{dmath}[]
		h^2 f[i]=u[i-1]+2 u[i]+u[i+1]
	\end{dmath}
\end{expl}
\section{Finite Elements}
\begin{expl}[1D problem]
	\begin{dgroup}[]
		\begin{dmath}[]
			-u''(x)=f(x)\condition*{x\in \Omega=(0,1)}
		\end{dmath}
		\begin{dmath}[compact]
			u(0)=u(1)=0
		\end{dmath}
		\begin{dsuspend}
			Assume PDF is satisfied \emph{pointwise}, then
		\end{dsuspend}
		\begin{dmath}[]
			-\int_{\Omega}^{}\dd{x}\,  u''(x)
			=\int_{\Omega}^{}\dd{x}\, f(x)
		\end{dmath}
		\begin{dsuspend}
			and also
		\end{dsuspend}
		\begin{dmath}[]
			-\int_{\Omega}^{} \dd{x}\, u''(x) \cdot v(x)
			=\int_{\Omega}^{}\dd{x}\,  f(x)\cdot v(x)
		\end{dmath}
		\begin{dsuspend}
			for arbitrary functions $v:\Omega \to \mathbb{R}$.
		\end{dsuspend}
	\end{dgroup}
\end{expl}
\begin{expl}
	\begin{itemize}
		\item Assume $v$ sufficiently smooth and 
			\begin{dmath}[compact]
				v(0)=v(1)=0
			\end{dmath}.
		\item Integration by parts
			\begin{dmath}[compact]
				\int_{a}^{b}\dd{x}\, f'(x) g(x)=[f(x)\cdot g(x)]_{a}^{b}-\int_{a}^{b}\dd{x}\, f(x) g'(x)
			\end{dmath}.
	\end{itemize}
	\begin{dgroup}[]
		\begin{dmath}[]
			-\int_{\Omega}^{} \dd{x}\, u''(x)\cdot v(x)
			=\int_{\Omega}^{}\dd{x}\, f(x)\cdot v(x)
		\end{dmath}
		\begin{dmath}[]
			\int_{\Omega}^{}\dd{x}\, u'(x)\cdot v'(x)
			=\int_{\Omega}^{}\dd{x}\, f(x)\cdot v(x)
			+[u'(x)\cdot v(x)]_{0}^{1}
		\end{dmath}
		\begin{dsuspend}
			with the imposed boundary conditions of $v(x)$ this leads to
		\end{dsuspend}
		\begin{dmath}[]
			\int_{\Omega}^{}\dd{x} \, u'(x) \cdot v'(x)
			=\int_{\Omega}^{}\dd{x}\, f(x) \cdot v(x)
		\end{dmath}
	\end{dgroup}
	\emph{Weak} form because of the \emph{weaker} continuity requirements.
\end{expl}
\subsection{Ritz-Galerkin Approach}
So far, $u(x)$ and $f(x)$ continuous function, but now choose finite-dimensional subspace for $u(x)$ and $f(x)$. Solve problem in \emph{weak} form in subspace (projection). $\to$ Discretize $u$
\begin{itemize}
	\item Sample with nodal positions $x_i$
	\item Nodal coefficients $u_i$
	\item Basis functions $N_i(x)$
\end{itemize}
\begin{dmath}[]
	\hiderel{\leadsto} u(x)=\sum_{i}^{}u_iN_i(x)
\end{dmath}
Discretize both $u$ and $v$ on $n$-dimensional space
\begin{dgroup*}[]
	\begin{dmath*}[]
		u(x)=\sum_{i=1}^{N}u_i N_i(x)
	\end{dmath*}
	\begin{dmath*}[]
		\hiderel{\leadsto} \pdv{}{x}u(x)=\sum_{i=1}^{n}u_i\pdv{}{x}N_i(x)
	\end{dmath*}
	\begin{dsuspend}
		insert into \emph{weak} formulation
	\end{dsuspend}
	\begin{dmath*}[]
		\int_{\Omega}^{} \dd{x}\, u'(x) v'(x)=\int_{\Omega}^{}\dd{x}\, f(x) v(x)
	\end{dmath*}
	\begin{dmath*}[]
		\hiderel{\leadsto} \int_{\Omega}^{}\dd{x}\,\sum_{i=1}^{n} u_i \pdv{N_i}{x}\cdot \sum_{j=1}^{n}v_j \pdv{N_j}{x}=\int_{\Omega}^{} \dd{x}\, f \cdot \sum_{j=1}^{n}v_j N_j(x)
	\end{dmath*}
	\begin{dmath*}[]
		\hiderel{\leadsto} \sum_{i,j\!=1}^{n}u_i v_j\int_{\Omega}^{}\dd{x}\, \pdv{N_i}{x} \pdv{N_j}{x}=\sum_{j=1}^{n} v_j\int_{\Omega}^{} \dd{x}\, f(x)  N_j(x)
	\end{dmath*}
\end{dgroup*}
\begin{dgroup*}[]
	\begin{dmath*}[]
		\hiderel{\leadsto} \sum_{j\!=1}^{n} v_j \!\left[ \sum_{i=1}^{n}u_i \int_{\Omega}^{}\dd{x} \, \pdv{N_i}{x}\pdv{N_j}{x}-\int_{\Omega}^{}\dd{x}\, f(x)N_j (x) \right]\!=0
	\end{dmath*},
	\begin{dsuspend}
		with an arbitrary $v_j$ (assumed)
	\end{dsuspend}
	\begin{dmath*}[]
		\sum_{i=1}^{n} u_i \int_{\Omega}^{}\dd{x}\, \pdv{N_i}{x}\pdv{N_j}{x}
		-\int_{\Omega}^{}\dd{x}\, f(x) N_i(x) =0 \condition*{\forall i,\ldots,n.}
	\end{dmath*}
\end{dgroup*}
Now we have $n$ linear equations for $n$ unknowns, which can be written as
\begin{dgroup}[]
	\begin{dmath}[]
		K \vtr{u}=\vtr{f}
	\end{dmath}
	\begin{dmath}[]
		\underbrace{
			\begin{pmatrix}
				K_{11}&\cdots &K_{1n}\\
				\vdots &\ddots & \vdots\\
				K_{n1}&\cdots & K_{nn}
			\end{pmatrix}\!
		}_{K}\,
		\underbrace{\!
			\begin{pmatrix}
				u_1\\
				\vdots\\
				u_n
			\end{pmatrix}\!
		}_{\vtr{u}}
		=
		\underbrace{\!
			\begin{pmatrix}
				f_1\\
				\vdots\\
				f_n
			\end{pmatrix}\!
		}_{\vtr{f}}
	\end{dmath}
\end{dgroup}

\begin{dgroup}[]
	\begin{dmath}[]
		K_{ij}=\int_{\Omega}^{} \dd{x}\, \pdv{N_i}{x}\pdv{N_j}{x}
	\end{dmath}
	\begin{dmath}[]
		f_i=\int_{\Omega}^{}\dd{x}\, f(x)N_i(x)
	\end{dmath}
\end{dgroup}

This matrix is 
\begin{itemize}
	\item symmetric (definition of $K_{ij}$)
	\item positive-definite (eliptic PDE)
	\item sparse ($N_i$ have compact support)
\end{itemize}
$\leadsto$ use sparse solvel, e.g., conjugate gradients
\paragraph{Choice of function space (types of Finite Elements)}
\begin{dmath}[]
	u(x)=\sum_{i}^{}u_i N_i(x)
\end{dmath}
\begin{itemize}
	\item smooth enough $\leadsto$ once differentiable
	\item simple $\leadsto$ polynomial functions
	\item interpolation $\leadsto$ $N_i(\vtr{x}_j)=\delta_{ij}$
	\item compact support $\leadsto$ defined piecewise on simple geometry
\end{itemize}
use piecewise linear basis functions $\leadsto$ piece wise linear approximation $u(x)$.
\paragraph{Linear simplicial elements}
Basis functions are uniquely defined through
\begin{itemize}
	\item Geometry $\vtr{x}_j$ and
	\item interpolations requirement $N_i(\vtr{x}_j)=\delta_{ij}$
\end{itemize}
\begin{dgroup}[]
	\begin{dmath}[]
		\vtr{x}_j=x_j \condition{in 1D,}
		=(x_j,y_j)^t \condition{in 2D,}
		=(x_j,y_j,z_j)^t \condition{in 3D}
	\end{dmath}.
\end{dgroup}
\begin{itemize}
	\item Basis functions are linear
		\begin{dmath}[]
			N_i(x,y)=a_i x+b_i y+ c
		\end{dmath}
	\item Due to $N_i(\vtr{x}_j)=\delta_{ij}$ we have
		\begin{dgroup}[]
			\begin{dmath}[]
				\begin{pmatrix}
					x_1&y_1&1\\
					x_2&y_2&1\\
					x_3&y_3&1
				\end{pmatrix}
				\begin{pmatrix}
					a_i\\
					b_i\\
					c_i
				\end{pmatrix}
				=
				\begin{pmatrix}
					\delta_{1i}\\
					\delta_{2i}\\
					\delta_{3i}
				\end{pmatrix}
			\end{dmath}
			\begin{dmath}[]
				\begin{pmatrix}
					a_i\\
					b_i\\
					c_i
				\end{pmatrix}
				=
				{\begin{pmatrix}
					x_1&y_1&1\\
					x_2&y_2&1\\
					x_3&y_3&1
				\end{pmatrix}}^{\!-1\!}
				\begin{pmatrix}
					\delta_{1i}\\
					\delta_{2i}\\
					\delta_{3i}
				\end{pmatrix}
			\end{dmath}
		\end{dgroup}
\end{itemize}
\paragraph{What is a finite element}
A finite element is a triple consisting of 
\begin{itemize}
	\item a closed subset $\Omega_{e}\subset \mathbb{R}^{d}$
	\item a set of $n$ basis function $N_i:\Omega_{E}\to \mathbb{R}$
	\item a set of $n$ associated nodal variables $\vtr{x}_i$
\end{itemize}
\paragraph{Summary of problem}
\begin{itemize}
	\item 1D Poisson problem given as
		\begin{gather}
			u''(x)=f(x)
		\end{gather}
	\item Weak form + galerkin approach gives linear system
		\begin{dgroup}[]
			\begin{dmath}[]
				K u = f
			\end{dmath}
			\begin{dsuspend}
				with
			\end{dsuspend}
			\begin{dmath}[]
				K_{ij}=\int_{\Omega}^{}\dd{x}\, \pdv{N_i}{x}\pdv{N_j}{x}
			\end{dmath},
			\begin{dsuspend}
				and
			\end{dsuspend}
			\begin{dmath}[]
				\vtr{f}_i=\int_{\Omega}^{}\dd{x}\, f(x)N_i(x)
			\end{dmath}.
		\end{dgroup}
	\item Use simple elements with linear basis functions
\end{itemize}
What is left is
\begin{itemize}
	\item tesseate domain into elements
	\item evaluate integrals (basis functions \& derivatives)
	\item assemble the system matrix and right hand side
\end{itemize}
\section{Exercise 2}
\paragraph{Task:} use linear triangle elements
\begin{itemize}
	\item evaluate integrals
	\item assemble the global system matrix and right hand side
\end{itemize}
\paragraph{Goal:} evaluate 
\begin{dmath}[]
	\vtr{f}_i=\int_{\Omega}^{}\dd{x}\dd{y}\, f(x)N_i(x)
\end{dmath}
\begin{itemize}
	\item $N_i$ extends ovel all $n_{e,i}$ elements incident to $\vtr{x}_i$
	\item $N_i$ is zero outside $\Omega_i\subset \Omega$
	\item Consider restrictions $N_{i}^{e}$ onto elements
\end{itemize}
It holds 
\begin{dgroup}[]
	\begin{dmath}[]
		\int_{\Omega}^{}\dd{x}\, f(x) N_i(x)
		=\sum_{\ell=1}^{n_{\ell,i}} \int_{\Omega_{\ell}}^{} \dd{x}\, f(x)N_{i}^{\ell}(x)\condition{with $\Omega_i =\bigcup_\ell \Omega_{\ell}$}
	\end{dmath}
	\begin{dsuspend}
		evaluate integrals over $\Omega_{\ell}$ with \emph{quadratic rule}
	\end{dsuspend}
	\begin{dmath}[]
		\int_{\Omega_{\ell}}^{} \rd x\rd y \, f(x)N_{i}^{e}(x)
		\approx f(x_{q},y_{q})\cdot N_{i}^{\ell} (x_{q},y_{q}) \cdot A_{\ell}
	\end{dmath},
	\begin{dsuspend}
		where $A_{\ell}$ area of element $\ell$ and $(x_{\ell},y_{\ell})$ the single quadrature point at bory-center
	\end{dsuspend}
\end{dgroup}
\paragraph{Goal:} evaluate
\begin{dgroup}[]
	\begin{dmath}[]
		\vtr{K}_{ij}=\int_{\Omega}^{}\rd x\rd y \, \int_{\Omega_{\ell}}^{}\rd x\rd y \, N_{i}^{\ell}N_{j}^{\ell} >0
	\end{dmath}
	\begin{dsuspend}
		i.e., $\exists$ an element containing vertices $\vtr{x}_i$ and $\vtr{x}_j$. Let $S_{ij}$ denote a set of all these elements, then
	\end{dsuspend}
	\begin{dmath}[]
		K_{ij}=\sum_{\ell \in S_{ij}}^{}\int_{\Omega_{\ell}}^{} \rd x\rd y \, \pdv{N_{i}^{\ell}}{x}\pdv{N_{j}^{\ell}}{x}+\pdv{N_{i}^{\ell}}{y}\pdv{N_{j}^{\ell}}{y}
	\end{dmath}.
\end{dgroup}
This is an element-centered implementation. For each element $\ell$, compute basis function derivatives
\begin{dgroup}[]
	\begin{dmath}[]
		\pdv{N_{k}^{\ell}}{\vtr{x}}\condition*{k=1,\ldots,3}
	\end{dmath}
	\begin{dsuspend}
		form products and integrate
	\end{dsuspend}
	\begin{dmath}[]
		\pdv{N_{k}^{\ell}}{\vtr{x}}\pdv{N_{m}^{\ell}}{\vtr{x}}\condition*{k=1,\ldots,3;\, m=k,\ldots, 3}
	\end{dmath},
	\begin{dsuspend}
		which is a constant on element and add to global matrix (local vs. global numbering)
	\end{dsuspend}
	\begin{dmath}[]
		K_{ij}\hiderel{+}=A_{\ell} \left( \pdv{N_{k}^{\ell}}{\vtr{x}}\pdv{N_{m}^{\ell}}{\vtr{x}} +\pdv{N_{k}^{\ell}}{\vtr{y}}\pdv{N_{m}^{\ell}}{\vtr{y}}\right)
	\end{dmath}
\end{dgroup}
\paragraph{Advantages of \emph{FEM}} (over Finite Differences)
\begin{itemize}
	\item can easily deal with comple geometries
	\item solution in nodes is naturally interpolated inside elements using basis functions
	\item weaker smoothness requirements on solution
	\item weak form is natural for many applications
\end{itemize}


\end{document}

% LocalWords:  asimple
