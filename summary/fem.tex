\chapter{Finite elements}
\section{Poisson's equation}
\begin{dgroup}[]
	\begin{dmath*}[]
		-\Delta u(x,y)=f(x,y)
	\end{dmath*}
\end{dgroup}
\section{Finite Difference}
\begin{expl}[1D problem]
	\begin{dmath}[]
		-u''(x)=f(x)\condition*{x\in \Omega=(0,1)}
	\end{dmath}
	\begin{dmath}[compact]
		u(0)=u(1)=0
	\end{dmath}
	\begin{dsuspend}
		regular grid
	\end{dsuspend}
	\begin{dmath}[]
		u[i]=u(i\cdot h)\condition*{i\in (0,\ldots,n)}
	\end{dmath}
	\begin{dsuspend}
		approximate derivatives
	\end{dsuspend}
	\begin{dmath}[]
		u''(x)=\frac{u'[i]-u'[i-1]}{h}=\frac{u[i+1]-2u[i]+u[i-1]}{h^2}
	\end{dmath}
	\begin{dsuspend}
		equation for eevry grid point $2,\cdots,n-1$
	\end{dsuspend}
	\begin{dmath}[]
		h^2 f[i]=u[i-1]+2 u[i]+u[i+1]
	\end{dmath}
\end{expl}
\section{Finite Elements}
\begin{expl}[1D problem]
	\begin{dgroup}[]
		\begin{dmath}[]
			-u''(x)=f(x)\condition*{x\in \Omega=(0,1)}
		\end{dmath}
		\begin{dmath}[compact]
			u(0)=u(1)=0
		\end{dmath}
		\begin{dsuspend}
			Assume PDF is satisfied \emph{pointwise}, then
		\end{dsuspend}
		\begin{dmath}[]
			-\int_{\Omega}^{}\dd{x}\,  u''(x)
			=\int_{\Omega}^{}\dd{x}\, f(x)
		\end{dmath}
		\begin{dsuspend}
			and also
		\end{dsuspend}
		\begin{dmath}[]
			-\int_{\Omega}^{} \dd{x}\, u''(x) \cdot v(x)
			=\int_{\Omega}^{}\dd{x}\,  f(x)\cdot v(x)
		\end{dmath}
		\begin{dsuspend}
			for arbitrary functions $v:\Omega \to \mathbb{R}$.
		\end{dsuspend}
	\end{dgroup}
\end{expl}
\begin{expl}
	\begin{itemize}
		\item Assume $v$ sufficiently smooth and 
			\begin{dmath}[compact]
				v(0)=v(1)=0
			\end{dmath}.
		\item Integration by parts
			\begin{dmath}[compact]
				\int_{a}^{b}\dd{x}\, f'(x) g(x)=[f(x)\cdot g(x)]_{a}^{b}-\int_{a}^{b}\dd{x}\, f(x) g'(x)
			\end{dmath}.
	\end{itemize}
	\begin{dgroup}[]
		\begin{dmath}[]
			-\int_{\Omega}^{} \dd{x}\, u''(x)\cdot v(x)
			=\int_{\Omega}^{}\dd{x}\, f(x)\cdot v(x)
		\end{dmath}
		\begin{dmath}[]
			\int_{\Omega}^{}\dd{x}\, u'(x)\cdot v'(x)
			=\int_{\Omega}^{}\dd{x}\, f(x)\cdot v(x)
			+[u'(x)\cdot v(x)]_{0}^{1}
		\end{dmath}
		\begin{dsuspend}
			with the imposed boundary conditions of $v(x)$ this leads to
		\end{dsuspend}
		\begin{dmath}[]
			\int_{\Omega}^{}\dd{x} \, u'(x) \cdot v'(x)
			=\int_{\Omega}^{}\dd{x}\, f(x) \cdot v(x)
		\end{dmath}
	\end{dgroup}
	\emph{Weak} form because of the \emph{weaker} continuity requirements.
\end{expl}
\subsection{Ritz-Galerkin Approach}
So far, $u(x)$ and $f(x)$ continuous function, but now choose finite-dimensional subspace for $u(x)$ and $f(x)$. Solve problem in \emph{weak} form in subspace (projection). $\to$ Discretize $u$
\begin{itemize}
	\item Sample with nodal positions $x_i$
	\item Nodal coefficients $u_i$
	\item Basis functions $N_i(x)$
\end{itemize}
\begin{dmath}[]
	\hiderel{\leadsto} u(x)=\sum_{i}^{}u_iN_i(x)
\end{dmath}
Discretize both $u$ and $v$ on $n$-dimensional space
\begin{dgroup*}[]
	\begin{dmath*}[]
		u(x)=\sum_{i=1}^{N}u_i N_i(x)
	\end{dmath*}
	\begin{dmath*}[]
		\hiderel{\leadsto} \pdv{}{x}u(x)=\sum_{i=1}^{n}u_i\pdv{}{x}N_i(x)
	\end{dmath*}
	\begin{dsuspend}
		insert into \emph{weak} formulation
	\end{dsuspend}
	\begin{dmath*}[]
		\int_{\Omega}^{} \dd{x}\, u'(x) v'(x)=\int_{\Omega}^{}\dd{x}\, f(x) v(x)
	\end{dmath*}
	\begin{dmath*}[]
		\hiderel{\leadsto} \int_{\Omega}^{}\dd{x}\,\sum_{i=1}^{n} u_i \pdv{N_i}{x}\cdot \sum_{j=1}^{n}v_j \pdv{N_j}{x}=\int_{\Omega}^{} \dd{x}\, f \cdot \sum_{j=1}^{n}v_j N_j(x)
	\end{dmath*}
	\begin{dmath*}[]
		\hiderel{\leadsto} \sum_{i,j\!=1}^{n}u_i v_j\int_{\Omega}^{}\dd{x}\, \pdv{N_i}{x} \pdv{N_j}{x}=\sum_{j=1}^{n} v_j\int_{\Omega}^{} \dd{x}\, f(x)  N_j(x)
	\end{dmath*}
\end{dgroup*}
\begin{dgroup*}[]
	\begin{dmath*}[]
		\hiderel{\leadsto} \sum_{j\!=1}^{n} v_j \!\left[ \sum_{i=1}^{n}u_i \int_{\Omega}^{}\dd{x} \, \pdv{N_i}{x}\pdv{N_j}{x}-\int_{\Omega}^{}\dd{x}\, f(x)N_j (x) \right]\!=0
	\end{dmath*},
	\begin{dsuspend}
		with an arbitrary $v_j$ (assumed)
	\end{dsuspend}
	\begin{dmath*}[]
		\sum_{i=1}^{n} u_i \int_{\Omega}^{}\dd{x}\, \pdv{N_i}{x}\pdv{N_j}{x}
		-\int_{\Omega}^{}\dd{x}\, f(x) N_i(x) =0 \condition*{\forall i,\ldots,n.}
	\end{dmath*}
\end{dgroup*}
Now we have $n$ linear equations for $n$ unknowns, which can be written as
\begin{dgroup}[]
	\begin{dmath}[]
		K \vtr{u}=\vtr{f}
	\end{dmath}
	\begin{dmath}[]
		\underbrace{
			\begin{pmatrix}
				K_{11}&\cdots &K_{1n}\\
				\vdots &\ddots & \vdots\\
				K_{n1}&\cdots & K_{nn}
			\end{pmatrix}\!
		}_{K}\,
		\underbrace{\!
			\begin{pmatrix}
				u_1\\
				\vdots\\
				u_n
			\end{pmatrix}\!
		}_{\vtr{u}}
		=
		\underbrace{\!
			\begin{pmatrix}
				f_1\\
				\vdots\\
				f_n
			\end{pmatrix}\!
		}_{\vtr{f}}
	\end{dmath}
\end{dgroup}

\begin{dgroup}[]
	\begin{dmath}[]
		K_{ij}=\int_{\Omega}^{} \dd{x}\, \pdv{N_i}{x}\pdv{N_j}{x}
	\end{dmath}
	\begin{dmath}[]
		f_i=\int_{\Omega}^{}\dd{x}\, f(x)N_i(x)
	\end{dmath}
\end{dgroup}

This matrix is 
\begin{itemize}
	\item symmetric (definition of $K_{ij}$)
	\item positive-definite (eliptic PDE)
	\item sparse ($N_i$ have compact support)
\end{itemize}
$\leadsto$ use sparse solvel, e.g., conjugate gradients
\paragraph{Choice of function space (types of Finite Elements)}
\begin{dmath}[]
	u(x)=\sum_{i}^{}u_i N_i(x)
\end{dmath}
\begin{itemize}
	\item smooth enough $\leadsto$ once differentiable
	\item simple $\leadsto$ polynomial functions
	\item interpolation $\leadsto$ $N_i(\vtr{x}_j)=\delta_{ij}$
	\item compact support $\leadsto$ defined piecewise on simple geometry
\end{itemize}
use piecewise linear basis functions $\leadsto$ piece wise linear approximation $u(x)$.
\paragraph{Linear simplicial elements}
Basis functions are uniquely defined through
\begin{itemize}
	\item Geometry $\vtr{x}_j$ and
	\item interpolations requirement $N_i(\vtr{x}_j)=\delta_{ij}$
\end{itemize}
\begin{dgroup}[]
	\begin{dmath}[]
		\vtr{x}_j=x_j \condition{in 1D,}
		=(x_j,y_j)^t \condition{in 2D,}
		=(x_j,y_j,z_j)^t \condition{in 3D}
	\end{dmath}.
\end{dgroup}
\begin{itemize}
	\item Basis functions are linear
		\begin{dmath}[]
			N_i(x,y)=a_i x+b_i y+ c
		\end{dmath}
	\item Due to $N_i(\vtr{x}_j)=\delta_{ij}$ we have
		\begin{dgroup}[]
			\begin{dmath}[]
				\begin{pmatrix}
					x_1&y_1&1\\
					x_2&y_2&1\\
					x_3&y_3&1
				\end{pmatrix}
				\begin{pmatrix}
					a_i\\
					b_i\\
					c_i
				\end{pmatrix}
				=
				\begin{pmatrix}
					\delta_{1i}\\
					\delta_{2i}\\
					\delta_{3i}
				\end{pmatrix}
			\end{dmath}
			\begin{dmath}[]
				\begin{pmatrix}
					a_i\\
					b_i\\
					c_i
				\end{pmatrix}
				=
				{\begin{pmatrix}
					x_1&y_1&1\\
					x_2&y_2&1\\
					x_3&y_3&1
				\end{pmatrix}}^{\!-1\!}
				\begin{pmatrix}
					\delta_{1i}\\
					\delta_{2i}\\
					\delta_{3i}
				\end{pmatrix}
			\end{dmath}
		\end{dgroup}
\end{itemize}
\paragraph{What is a finite element}
A finite element is a triple consisting of 
\begin{itemize}
	\item a closed subset $\Omega_{e}\subset \mathbb{R}^{d}$
	\item a set of $n$ basis function $N_i:\Omega_{E}\to \mathbb{R}$
	\item a set of $n$ associated nodal variables $\vtr{x}_i$
\end{itemize}
\paragraph{Summary of problem}
\begin{itemize}
	\item 1D Poisson problem given as
		\begin{gather}
			u''(x)=f(x)
		\end{gather}
	\item Weak form + galerkin approach gives linear system
		\begin{dgroup}[]
			\begin{dmath}[]
				K u = f
			\end{dmath}
			\begin{dsuspend}
				with
			\end{dsuspend}
			\begin{dmath}[]
				K_{ij}=\int_{\Omega}^{}\dd{x}\, \pdv{N_i}{x}\pdv{N_j}{x}
			\end{dmath},
			\begin{dsuspend}
				and
			\end{dsuspend}
			\begin{dmath}[]
				\vtr{f}_i=\int_{\Omega}^{}\dd{x}\, f(x)N_i(x)
			\end{dmath}.
		\end{dgroup}
	\item Use simple elements with linear basis functions
\end{itemize}
What is left is
\begin{itemize}
	\item tesseate domain into elements
	\item evaluate integrals (basis functions \& derivatives)
	\item assemble the system matrix and right hand side
\end{itemize}
\section{Exercise 2}
\paragraph{Task:} use linear triangle elements
\begin{itemize}
	\item evaluate integrals
	\item assemble the global system matrix and right hand side
\end{itemize}
\paragraph{Goal:} evaluate 
\begin{dmath}[]
	\vtr{f}_i=\int_{\Omega}^{}\dd{x}\dd{y}\, f(x)N_i(x)
\end{dmath}
\begin{itemize}
	\item $N_i$ extends ovel all $n_{e,i}$ elements incident to $\vtr{x}_i$
	\item $N_i$ is zero outside $\Omega_i\subset \Omega$
	\item Consider restrictions $N_{i}^{e}$ onto elements
\end{itemize}
It holds 
\begin{dgroup}[]
	\begin{dmath}[]
		\int_{\Omega}^{}\dd{x}\, f(x) N_i(x)
		=\sum_{\ell=1}^{n_{\ell,i}} \int_{\Omega_{\ell}}^{} \dd{x}\, f(x)N_{i}^{\ell}(x)\condition{with $\Omega_i =\bigcup_\ell \Omega_{\ell}$}
	\end{dmath}
	\begin{dsuspend}
		evaluate integrals over $\Omega_{\ell}$ with \emph{quadratic rule}
	\end{dsuspend}
	\begin{dmath}[]
		\int_{\Omega_{\ell}}^{} \rd x\rd y \, f(x)N_{i}^{e}(x)
		\approx f(x_{q},y_{q})\cdot N_{i}^{\ell} (x_{q},y_{q}) \cdot A_{\ell}
	\end{dmath},
	\begin{dsuspend}
		where $A_{\ell}$ area of element $\ell$ and $(x_{\ell},y_{\ell})$ the single quadrature point at bory-center
	\end{dsuspend}
\end{dgroup}
\paragraph{Goal:} evaluate
\begin{dgroup}[]
	\begin{dmath}[]
		\vtr{K}_{ij}=\int_{\Omega}^{}\rd x\rd y \, \int_{\Omega_{\ell}}^{}\rd x\rd y \, N_{i}^{\ell}N_{j}^{\ell} >0
	\end{dmath}
	\begin{dsuspend}
		i.e., $\exists$ an element containing vertices $\vtr{x}_i$ and $\vtr{x}_j$. Let $S_{ij}$ denote a set of all these elements, then
	\end{dsuspend}
	\begin{dmath}[]
		K_{ij}=\sum_{\ell \in S_{ij}}^{}\int_{\Omega_{\ell}}^{} \rd x\rd y \, \pdv{N_{i}^{\ell}}{x}\pdv{N_{j}^{\ell}}{x}+\pdv{N_{i}^{\ell}}{y}\pdv{N_{j}^{\ell}}{y}
	\end{dmath}.
\end{dgroup}
This is an element-centered implementation. For each element $\ell$, compute basis function derivatives
\begin{dgroup}[]
	\begin{dmath}[]
		\pdv{N_{k}^{\ell}}{\vtr{x}}\condition*{k=1,\ldots,3}
	\end{dmath}
	\begin{dsuspend}
		form products and integrate
	\end{dsuspend}
	\begin{dmath}[]
		\pdv{N_{k}^{\ell}}{\vtr{x}}\pdv{N_{m}^{\ell}}{\vtr{x}}\condition*{k=1,\ldots,3;\, m=k,\ldots, 3}
	\end{dmath},
	\begin{dsuspend}
		which is a constant on element and add to global matrix (local vs. global numbering)
	\end{dsuspend}
	\begin{dmath}[]
		K_{ij}\hiderel{+}=A_{\ell} \left( \pdv{N_{k}^{\ell}}{\vtr{x}}\pdv{N_{m}^{\ell}}{\vtr{x}} +\pdv{N_{k}^{\ell}}{\vtr{y}}\pdv{N_{m}^{\ell}}{\vtr{y}}\right)
	\end{dmath}
\end{dgroup}
\paragraph{Advantages of \emph{FEM}} (over Finite Differences)
\begin{itemize}
	\item can easily deal with comple geometries
	\item solution in nodes is naturally interpolated inside elements using basis functions
	\item weaker smoothness requirements on solution
	\item weak form is natural for many applications
\end{itemize}
