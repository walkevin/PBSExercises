\begin{frame}
	\titlepage
\end{frame}
\begin{frame}
	\tableofcontents%[hideallsubsections]
\end{frame}
\section{Introduction}
\begin{frame}
	\begin{itemize}
		\item Initial project idea:\\
			Paintball shot on wall
		\item Final project:\\
			Paintball shot on simple objects
	\end{itemize}
\end{frame}
\subsection{Inspiration}
\tikzsetnextfilename{video}
\begin{frame}
	\begin{center}
		\tikz\node[fill=rioday,minimum width=5cm,minimum height=2cm,text=black,rounded corners=10pt] (a) at (0,0) {\LARGE Video};
	\end{center}
\end{frame}
\begin{frame}
		\hspace*{-1cm}\movie[height=\textheight,width=\paperwidth,poster,showcontrols=true]{}{paintball-def.mp4}
\end{frame}
\subsection{Expected challenges}
\begin{frame}
	\frametitle{Expected challenges}
	\begin{itemize}
		\item OpenGL 
			\begin{itemize}
				\item showing simple objects
			\end{itemize}
		\item SPH solver
			\begin{itemize}
				\item time to make it work as expected
			\end{itemize}
	\end{itemize}
\end{frame}
\section{Approach}
\begin{frame}
	\only<1> {\frametitle{Approach}}
	\only<2> {\frametitle{Why?}}
	\begin{itemize}[<+->]
		\item<1-> SPH solver
			\begin{itemize}
				\item<2-> splashes and droplets
			\end{itemize}
		\item Collision handling with no library but boundary boxes
			\begin{itemize}
					\item<2-> late attacket
					\item<2-> risk of heavy time investission
			\end{itemize}
	\end{itemize}
\end{frame}
\section{Implementation}
\subsection{Low level}
\begin{frame}
	\frametitle{Low level}
	\begin{itemize}
		\item C++ 11
			\begin{itemize}
				\item eigen
			\end{itemize}
		\item OpenGL
			\begin{itemize}
				\item glsl
				\item glm
			\end{itemize}
	\end{itemize}
\end{frame}
\subsection{High level}
\begin{frame}
	\frametitle{High level}
	\begin{enumerate}
		\item Shot with gravitation
		\item Collision detection
		\item On impact the SPH simulation starts
	\end{enumerate}
\end{frame}
\subsection{Difficulties}
\begin{frame}
	\frametitle{Difficulties}
	\begin{itemize}
		\item (general code debugging)
		\item (parameter tuning)
		\item (coordinates: simulation space $\leftrightarrow$ window space)
		\item many particles behaving like a ball
		\item particles pushed and trapped inside object
		\item sticky paint on object with given resolution
	\end{itemize}
\end{frame}
\subsection{Tricks}
\begin{frame}
	\frametitle{Tricks}
	\begin{itemize}
		\item many particles behaving like a ball
			\begin{itemize}
				\item[$\to$] particle-particle forces ignored before collision
			\end{itemize}
		\item particles pushed and trapped inside object
			\begin{itemize}
				\item[$\to$] projection further than on surface
			\end{itemize}
		\item sticky paint on object
			\begin{itemize}
				\item[$\to$] velocity dependent trace on object
			\end{itemize}
	\end{itemize}
\end{frame}
\begin{frame}
	\frametitle{Particles pushed and trapped inside object}
	\begin{columns}
		\begin{column}{0.5\textwidth}
			\begin{figure}[]
					\tikzsetnextfilename{pp1}
					\only<1>{
						\begin{center}
							\begin{tikzpicture}[every node/.style={pin distance=0.5ex},every pin edge/.style={draw=black}]
	\def\circlesize{0.5cm}
	\def\particlesize{0.3cm}
		\fill[white!70!black] (0,2) rectangle (2.5,-1);
		\node[fill=rioday,rounded corners=1ex] (tl) at (0,3.5) {naive method};
		\draw[ultra thick] (0,2) -- (0,-1);
		\node[fill=rioday,circle,minimum size=\particlesize,pin=above:$t^{(n)}$] (tn) at (-1,2) {};
		\node[fill=rioday,circle,minimum size=\particlesize,pin=above right:$t^{(n+1)}$] (tnp1) at (1,0) {};
		\node[fill=rioday,circle,minimum size=\particlesize,pin=below right:$t_{\text{corr}}^{(n+1)}$] (tnp1corr) at (0,0) {};
		\node[circle,draw,dashed,minimum size=\circlesize] (col1) at (0,1) {};
		\draw[-latex,thick] (tn) -- (tnp1);
		\path[draw,-latex] (tnp1) to[above] node{$\ell$} (tnp1corr);
		\foreach \x/\y/\z in {%
			1/-1/-0.5,
			2/-1.2/-0.1,
			3/-0.3/0.3,
			4/-0.5/-0.25,
			5/-1.3/0.8,
			6/-1.1/0.3,
			7/-0.7/0.1
		}
			\node[circle,fill=white,minimum size=\particlesize] (p\z) at (\y,\z) {};
\end{tikzpicture}

						\end{center}
					}
					\tikzsetnextfilename{pp2}
					\only<2>{
						\begin{center}
							\begin{tikzpicture}[every node/.style={pin distance=0.5ex},every pin edge/.style={draw=black}]
	\def\circlesize{0.5cm}
	\def\particlesize{0.3cm}
		\fill[white!70!black] (0,2) rectangle (2.5,-1);
		\node[fill=rioday,rounded corners=1ex] (tl) at (0,3.5) {naive method};
		\draw[ultra thick] (0,2) -- (0,-1);
		\node[fill=rioday,circle,minimum size=\particlesize,pin=above:$t^{(n)}$] (tn) at (-1,2) {};
		\node[fill=rioday,circle,minimum size=\particlesize,pin=above right:$t^{(n+1)}$] (tnp1) at (1,0) {};
		\node[fill=rioday,circle,minimum size=\particlesize,pin=below right:$t_{\text{corr}}^{(n+1)}$] (tnp1corr) at (0,0) {};
		\node[circle,draw,dashed,minimum size=\circlesize] (col1) at (0,1) {};
		\draw[-latex,thick] (tn) -- (tnp1);
		\path[draw,-latex] (tnp1) to[above] node{$\ell$} (tnp1corr);
		\foreach \x/\y/\z in {%
			1/-1/-0.5,
			2/-1.2/-0.1,
			3/-0.3/0.3,
			4/-0.5/-0.25,
			5/-1.3/0.8,
			6/-1.1/0.3,
			7/-0.7/0.1
		}
			\node[circle,fill=helsinkinight,minimum size=\particlesize] (p\z) at (\y,\z) {};
\end{tikzpicture}

						\end{center}
					}
					\tikzsetnextfilename{pp3}
					\only<3>{
						\begin{center}
							\begin{tikzpicture}[every node/.style={pin distance=0.5ex},every pin edge/.style={draw=black}]
	\def\circlesize{0.5cm}
	\def\particlesize{0.3cm}
		\fill[white!70!black] (0,2) rectangle (2.5,-1);
		\node[fill=rioday,rounded corners=1ex] (tl) at (0,3.5) {naive method};
		\draw[ultra thick] (0,2) -- (0,-1);
		\node[fill=rioday,circle,minimum size=\particlesize,pin=above:$t^{(n)}$] (tn) at (-1,2) {};
		\node[fill=rioday,circle,minimum size=\particlesize,pin=above right:$t^{(n+1)}$] (tnp1) at (1,0) {};
		\node[fill=rioday,circle,minimum size=\particlesize,pin=below right:$t_{\text{corr}}^{(n+1)}$] (tnp1corr) at (0,0) {};
		\node[circle,draw,dashed,minimum size=\circlesize] (col1) at (0,1) {};
		\draw[-latex,thick] (tn) -- (tnp1);
		\path[draw,-latex] (tnp1) to[above] node{$\ell$} (tnp1corr);
		\foreach \x/\y/\z in {%
			1/-1/-0.5,
			2/-1.2/-0.1,
			3/-0.3/0.3,
			4/-0.5/-0.25,
			5/-1.3/0.8,
			6/-1.1/0.3,
			7/-0.7/0.1
		}
			\node[circle,fill=helsinkinight,minimum size=\particlesize] (p\z) at (\y,\z) {};
\end{tikzpicture}

						\end{center}
					}
				\label{fig:pm1}
			\end{figure}
		\end{column}
		\begin{column}{0.5\textwidth}
			\begin{figure}[]
				\uncover<3>{
						\begin{center}
							\tikzsetnextfilename{pp4}
							\begin{tikzpicture}[every node/.style={pin distance=0.5ex},every pin edge/.style={draw=black}]
	\def\circlesize{0.5cm}
	\def\particlesize{0.3cm}
		\fill[white!70!black] (0,2) rectangle (2.5,-1);
		\node[fill=rioday,rounded corners=1ex] (tl) at (0,3.5) {our method};
		\draw[ultra thick] (0,2) -- (0,-1);
		\node[fill=rioday,circle,minimum size=\particlesize,pin=above:$t^{(n)}$] (tn) at (-1,2) {};
		\node[fill=rioday,circle,minimum size=\particlesize,pin=above right:$t^{(n+1)}$] (tnp1) at (1,0) {};
		\node[fill=rioday,circle,minimum size=\particlesize,pin=below right:$t_{\text{corr}}^{(n+1)}$] (tnp1corr) at (-3,0) {};
		\node[circle,draw,dashed,minimum size=\circlesize] (col1) at (0,1) {};
		\draw[-latex,thick] (tn) -- (tnp1);
		\path[draw,-latex] (tnp1) to[above,pos=0.8] node{$4\ell$} (tnp1corr);
		\foreach \x/\y/\z in {%
			1/-1/-0.5,
			2/-1.2/-0.1,
			3/-0.3/0.3,
			4/-0.5/-0.25,
			5/-1.3/0.8,
			6/-1.1/0.3,
			7/-0.7/0.1
		}
			\node[circle,fill=helsinkinight,minimum size=\particlesize] (p\z) at (\y,\z) {};
\end{tikzpicture}

						\end{center}
				\label{fig:p2}
				}
			\end{figure}
		\end{column}
	\end{columns}
\end{frame}
\begin{frame}
	\frametitle{Sticky paint on object}
	\framesubtitle{Trace handling}
\tikzsetnextfilename{collision-trace}
			\begin{figure}[]
				\begin{center}
					\begin{tikzpicture}[fill=rioday,every node/.style={label distance=0.5cm}]
						\begin{scope}[shift={(-5.5,0)},fill=black,every node/.style={label distance=1ex}]
							\node[fill=rioday,rounded corners=1ex] (tl) at (-1,3) {Velocity split};
							\node[circle,fill=helsinkinight,minimum size=0.5cm,pin=above:SPH particle] (cb) at (-1,1) {};
							\draw[very thick] (0,-1) -- (0,1.5);
							\coordinate[] (sf) at (0,0);
							\coordinate[] (sfe) at (-2,0);
							\path[-latex,draw] (cb) to[above] node{$\vtr{v}$} (sf);
							\path[-latex,draw] (cb) to[above] node{$\vtr{v}_{\perp}$} +(1,0);
							\path[-latex,draw] (cb) to[left] node{$\vtr{v}_{\parallel}$} +(0,-1);
						\end{scope}
						\node[fill=rioday,rounded corners=1ex] (tl) at (0,3) {Resulting trace};
						\node[circle,fill,minimum size=1cm,label={[align=center]left:{$\vtr{v}_{\parallel}=0$\\$\abs{\vtr{v}}$ big}}] (c1) at (-1,1) {};
						\node[ellipse,fill,rotate=45,minimum width=2cm,minimum height=0.5cm,label={[align=center]below right:{$\vtr{v}_{\parallel}\neq0$\\$\abs{\vtr{v}}$ big}}] (e1) at (1,1) {};
						\node[circle,fill,minimum size=0.4cm,label={[align=center]left:{$\vtr{v}_{\parallel}=0$\\$\abs{\vtr{v}}$ small}}] (c2) at (-1,-1) {};
						\node[ellipse,fill,rotate=45,minimum width=0.8cm,minimum height=0.2cm,label={[align=center]below right:{$\vtr{v}_{\parallel}\neq0$\\$\abs{\vtr{v}}$ small}}] (e2) at (1,-1) {};
					\end{tikzpicture}
				\end{center}
				\label{fig:velocity}
			\end{figure}
\end{frame}
\section{Demonstration}
\begin{frame}
\tikzsetnextfilename{demo}
	\begin{center}
		\tikz\node[fill=rioday,minimum width=5cm,minimum height=2cm,text=black,rounded corners=10pt] (a) at (0,0) {\LARGE Demonstration};
	\end{center}
\end{frame}
\begin{frame}
	\begin{columns}[t]
		\begin{column}{0.5\textwidth}
			\begin{block}{Performance}
				\begin{itemize}
					\item Real time simulation for simple objects (500 vertices)
					\item For complex objects substantionally slower
				\end{itemize}
			\end{block}
		\end{column}
		\pause
		\begin{column}{0.5\textwidth}
			\begin{block}{Limitations}
				\begin{itemize}
					\item resolution and number of collision objects
						\begin{itemize}
							\item[$\to$] bottleneck collision handling
						\end{itemize}
					\item number of particles
						\begin{itemize}
							\item[$\to$] bottleneck SPH simulation
						\end{itemize}
				\end{itemize}
			\end{block}
		\end{column}
	\end{columns}
\end{frame}
\section{Conclusion}
\begin{frame}
	\begin{columns}[t]
		\begin{column}{0.5\textwidth}
			\begin{block}{What we learned}
				\begin{itemize}
					\item OpenGL
					\item Blender
				\end{itemize}
			\end{block}
		\end{column}
		\pause
		\begin{column}{0.5\textwidth}
			\begin{block}{What we would do different}
				\begin{itemize}
					\item Start collision handling earlier
				\end{itemize}
			\end{block}
		\end{column}
	\end{columns}
\end{frame}
