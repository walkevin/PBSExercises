\begin{frame}
	\titlepage
\end{frame}
\begin{frame}
	\tableofcontents%[hideallsubsections]
\end{frame}
\section{Introduction}
\begin{frame}
	\begin{itemize}
		\item Initial project idea:\\
			Paintball shot on wall
		\item Final project:\\
			Paintball shot on simple objects
	\end{itemize}
\end{frame}
\subsection{Inspiration}
\begin{frame}
	\begin{center}
		\tikz\node[fill=rioday,minimum width=5cm,minimum height=2cm,text=black,rounded corners=10pt] (a) at (0,0) {\LARGE Video};
	\end{center}
\end{frame}
\begin{frame}
		\hspace*{-1cm}\movie[height=\textheight,width=\paperwidth,poster,showcontrols=true]{}{paintball-def.mp4}
\end{frame}
\subsection{Expected challenges}
\begin{frame}
	\frametitle{Expected challenges}
	\begin{itemize}
		\item OpenGL 
			\begin{itemize}
				\item showing simple objects
			\end{itemize}
		\item SPH solver
			\begin{itemize}
				\item time to make it work as expected
			\end{itemize}
	\end{itemize}
\end{frame}
\section{Approach}
\begin{frame}
	\only<1> {\frametitle{Approach}}
	\only<2> {\frametitle{Why?}}
	\begin{itemize}
		\item SPH solver
			\begin{itemize}
%				\only<1>{\transparent{0}}
				\only<1>{\item[]}
				\only<2>{\item splashes and droplets}
			\end{itemize}
		\item Collision handling with no library but boundary boxes
			\begin{itemize}
%				\only<1>{\transparent{0}}
				\only<1>{
					\item[]
					\item[]
				}
				\only<2>{
					\item late attacket
					\item risk of heavy time investission
				}
			\end{itemize}
	\end{itemize}
\end{frame}
\section{Implementation}
\subsection{Low level}
\begin{frame}
	\frametitle{Low level}
	\begin{itemize}
		\item C++ 11
			\begin{itemize}
				\item eigen
			\end{itemize}
		\item OpenGL
			\begin{itemize}
				\item glsl
				\item glm
			\end{itemize}
	\end{itemize}
\end{frame}
\subsection{High level}
\begin{frame}
	\frametitle{High level}
	\begin{enumerate}
		\item Shot with gravitation
		\item Collision detection
		\item On impact the SPH simulation starts
	\end{enumerate}
\end{frame}
\subsection{Difficulties}
\begin{frame}
	\frametitle{Difficulties}
	\begin{itemize}
		\item Debugging
		\item general code
		\item parameter tuning
		\item coordinates: simulation space $\leftrightarrow$ window space
		\item many particles behaving like a ball
		\item sticky paint on object with given resolution
	\end{itemize}
\end{frame}
\subsection{Tricks}
\begin{frame}
	\frametitle{Tricks}
	\begin{itemize}
		\item many particles behaving like a ball
			\begin{itemize}
				\item particle-particle forces ignored before collision
			\end{itemize}
		\item sticky paint on object
			\begin{itemize}
				\item velocity dependent trace on object
					\begin{gather*}
						\vtr{v}=\vtr{v}_{\parallel}+\vtr{v}_{\perp}.
					\end{gather*}
				\item ''$\parallel$`` and ''$\perp$`` describe angle between particle velocity and surface
				\item set $v_{\perp}=0$ after collision
			\end{itemize}
	\end{itemize}
\end{frame}
\begin{frame}
	\frametitle{Sticky paint on object}
	\framesubtitle{Trace handling}
\tikzsetnextfilename{collision-trace}
			\begin{figure}[]
				\begin{center}
					\begin{tikzpicture}[fill=rioday,every node/.style={label distance=0.5cm}]
						\begin{scope}[shift={(-5.5,0)},fill=black,every node/.style={label distance=1ex}]
							\node[fill=rioday,rounded corners=1ex] (tl) at (-1,3) {Velocity split};
							\node[circle,fill=helsinkinight,minimum size=0.5cm,pin=above:SPH particle] (cb) at (-1,1) {};
							\draw[very thick] (0,-1) -- (0,1.5);
							\coordinate[] (sf) at (0,0);
							\coordinate[] (sfe) at (-2,0);
							\path[-latex,draw] (cb) to[above] node{$\vtr{v}$} (sf);
							\path[-latex,draw] (cb) to[above] node{$\vtr{v}_{\perp}$} +(1,0);
							\path[-latex,draw] (cb) to[left] node{$\vtr{v}_{\parallel}$} +(0,-1);
						\end{scope}
						\node[fill=rioday,rounded corners=1ex] (tl) at (0,3) {Resulting trace};
						\node[circle,fill,minimum size=1cm,label={[align=center]left:{$\vtr{v}_{\parallel}=0$\\$\abs{\vtr{v}}$ big}}] (c1) at (-1,1) {};
						\node[ellipse,fill,rotate=45,minimum width=2cm,minimum height=0.5cm,label={[align=center]below right:{$\vtr{v}_{\parallel}\neq0$\\$\abs{\vtr{v}}$ big}}] (e1) at (1,1) {};
						\node[circle,fill,minimum size=0.4cm,label={[align=center]left:{$\vtr{v}_{\parallel}=0$\\$\abs{\vtr{v}}$ small}}] (c2) at (-1,-1) {};
						\node[ellipse,fill,rotate=45,minimum width=0.8cm,minimum height=0.2cm,label={[align=center]below right:{$\vtr{v}_{\parallel}\neq0$\\$\abs{\vtr{v}}$ small}}] (e2) at (1,-1) {};
					\end{tikzpicture}
				\end{center}
				\label{fig:velocity}
			\end{figure}
\end{frame}
\section{Demonstration}
\begin{frame}
	\begin{center}
		\tikz\node[fill=rioday,minimum width=5cm,minimum height=2cm,text=black,rounded corners=10pt] (a) at (0,0) {\LARGE Demonstration};
	\end{center}
\end{frame}
\begin{frame}
	\begin{columns}[t]
		\begin{column}{0.5\textwidth}
			\begin{block}{Performance}
				\begin{itemize}
					\item Real time simulation for simple objects (500 vertices)
				\end{itemize}
			\end{block}
		\end{column}
		\begin{column}{0.5\textwidth}
			\begin{block}{Limitations}
				\begin{itemize}
					\item No complex objects because of simple collision handling
				\end{itemize}
			\end{block}
		\end{column}
	\end{columns}
\end{frame}
\section{Conclusion}
\begin{frame}
	\begin{columns}[t]
		\begin{column}{0.5\textwidth}
			\begin{block}{What we learned}
				\begin{itemize}
					\item OpenGL
					\item Blend
				\end{itemize}
			\end{block}
		\end{column}
		\begin{column}{0.5\textwidth}
			\begin{block}{What we would do different}
				\begin{itemize}
					\item Start collision handling earlier
				\end{itemize}
			\end{block}
		\end{column}
	\end{columns}
\end{frame}
