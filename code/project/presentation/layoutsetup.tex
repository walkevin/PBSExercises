%% Memoir layout setup

%% NOTE: You are strongly advised not to change any of them unless you
%% know what you are doing.  These settings strongly interact in the
%% final look of the document.

% Dependencies
\usepackage{array}
\usetheme{Copenhagen}
\useinnertheme{circles}

% Define the default sans serif font as the lighter computer modern bright by
% D. Knuth.
%%% Define a nice orange color and use it for hyperref
%%% Needs xcolor
\usepackage{xcolor}
\definecolor{rioday}{RGB}{255,166,0} 
\hypersetup{colorlinks=false,linkbordercolor=rioday}

%\setbeamercolor{background canvas}{bg=black}
\setbeamercolor{frametitle}{fg=black,bg=rioday}
\setbeamercolor{structure}{fg=helsinkinight!60!white}
%\setbeamercolor{alerted text}{fg=helsinkinight,bg=white}
%\setbeamercolor{example text}{fg=helsinkinight,bg=white}
\setbeamercolor{item}{fg=helsinkinight}
\setbeamercolor{item projected}{fg=black,bg=rioday}
\setbeamercolor{subitem projected}{fg=white,bg=helsinkinight}
%\setbeamercolor*{enumerate item}{fg=rioday}
%\setbeamercolor*{enumerate subitem}{fg=helsinkinight}
%\setbeamertemplate{enumerate items}[circles]
%\setbeamertemplate{enumerate subitems}[squares]
\setbeamercolor{section in head/foot}{fg=white,bg=helsinkinight}
%\setbeamercolor{inner color}{fg=rioday,bg=helsinkinight}
%\setbeamercolor{outer color}{fg=rioday,bg=helsinkinight}

\setbeamercolor{palette primary}{fg=black,bg=rioday}
%\setbeamercolor{normal text}{fg=white,bg=black}


% This defines how theorems should look. Best leave as is.
%\theoremstyle{plain}
%\theoremseparator{:\quad}
%\theoremprework{}
%\theoremindent2em
%\theoremheaderfont{\sffamily\bfseries}
%\theorembodyfont{\normalfont}
%%\theoremsymbol{}
%%% Minimal margin to print two pages on an A4 paper or viewing it on tablets.
%\settypeblocksize{17.7cm}{11.8cm}{*}
%\setlrmargins{2cm}{*}{*}
%\setulmargins{1.6cm}{*}{*}
%\setheadfoot{7pt}{20pt}
%\setlength{\beforechapskip}{-1.2cm}
%\checkandfixthelayout
