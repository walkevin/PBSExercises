\usetikzlibrary{positioning,calc,matrix,chains,scopes,fit,decorations,decorations.pathmorphing,decorations.pathreplacing,arrows,patterns,3d}
\usepgfplotslibrary{external}
\tikzexternalize[prefix=tikz-pictures/]

\definecolor{rioday}{RGB}{255,166,0} 
\colorlet{riomorning}{rioday!65!black}
\colorlet{rionight}{rioday!50!black}

\definecolor{helsinkiday}{RGB}{0,89,255} 
%%% Hi from 2013. I think it's fascinating that this code could possibly be used until like 2043 or even further.
\colorlet{helsinkimorning}{helsinkiday!65!black}
\colorlet{helsinkinight}{helsinkiday!50!black}
\hypersetup{colorlinks=false,linkbordercolor=rioday}
%%%This is the global setting of the number of samples used to plot the functions in the graphs. 
\tikzset{
	smasa/.style={
		%%%The number of samples with low number.
		%%% 1000 should be enough for a final edition
		%samples=10
		samples=1000
	},
	bigsa/.style={
		%%%The number of samples with high number.
		%%% 10000 should be enough for a final edition.
		%samples=20
		samples=10000
	},
	%%%The standard filling
	sfill/.style={
		fill=rioday
	},
	zyplane/.style={canvas is zy plane at x=#1,very thin},
	zxplane/.style={canvas is zx plane at y=#1,very thin},
	yxplane/.style={canvas is yx plane at z=#1,very thin},
	bigspinmatrix/.style={ampersand replacement=\&,column sep=0pt,row sep=1em},
	hugespinmatrix/.style={ampersand replacement=\&,column sep=0pt,row sep=1em,font=\Large},
	bigspin/.style={minimum size=1.5em,rectangle,rounded corners=1ex},
	hugespin/.style={minimum size=2.25em,rectangle,rounded corners=1.5ex},
	bigspinup/.style={bigspin,fill=rioday},
	bigspindown/.style={bigspin,fill=helsinkinight,text=white},
	hugespinup/.style={hugespin,fill=rioday},
	hugespindown/.style={hugespin,fill=helsinkinight,text=white},
	smallspin/.style={rectangle,minimum size=0.32cm,draw,very thick},
	smallspinup/.style={smallspin,fill=rioday},
	smallspindown/.style={smallspin,fill=helsinkinight},
	matrixcircle/.style={circle,minimum size=2.0cm}
}
\newcommand{\scu}[2]{\node[smallspinup,fill={rioday!#1!red},opacity=#2]{};}
\newcommand{\scd}[2]{\node[smallspindown,fill={helsinkinight!#1!red},opacity=#2]{};}
\newcommand{\bcu}[2]{\node[bigspinup,fill={rioday!#1!red},opacity=#2]{$\uparrow$};}
\newcommand{\bcd}[2]{\node[bigspindown,fill={helsinkinight!#1!red},opacity=#2]{$\downarrow$};}
\newcommand{\hcun}[3]{\node[hugespinup,fill={rioday!#1!red},opacity=#2](#3){$\uparrow$};}
\newcommand{\hcdn}[3]{\node[hugespindown,fill={helsinkinight!#1!red},opacity=#2](#3){$\downarrow$};}
