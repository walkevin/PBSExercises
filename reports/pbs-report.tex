%%%\documentclass[paper,notoc,12pt]{JHEP3}
%%%mmaetz: put a4paper
%\documentclass[a5paper]{book}
\documentclass[10pt,a5paper,article,twoside]{memoir}
%%%mmaetz: reduce margins
%%%mmaetz: omitted temporarily to have the margins for the change remarks.
%\usepackage[a5paper,left=2cm, right=1cm,top=1.5cm, bottom=1.5cm]{geometry}
%mmaetz: use tikz package
\usepackage{tikz,tikz-3dplot,pgflibraryshapes}
%\usepackage[headings,cm]{fullpage}
\usetikzlibrary{positioning,calc,matrix,chains,scopes,fit,decorations,decorations.pathmorphing,decorations.pathreplacing,arrows,patterns,3d}
%%%\usepackage{pgfplots}
\usepackage[utf8]{inputenc}
\usepackage[ngerman,english]{babel}

%\usepackage{epsfig,cite,amsmath,amssymb,amsbsy,mathrsfs}
\usepackage{epsfig,cite,amsmath,amssymb,amsbsy,mathrsfs}
\usepackage{graphicx}
\usepackage{color}
\usepackage{latexsym}
%%% mmaetz packages
%%% Side captions
\usepackage{sidecap}
%%% Theorems like Theorem 3.1, proof environment
%\usepackage{amsthm}
%% NTheorem is a reimplementation of the AMS Theorem package. This will allow
%% us to typeset theorems like examples, proofs and similar.
%% NOTE: Must be loaded AFTER amsmath, or the \qed placement will break
\usepackage[amsthm,thmmarks]{ntheorem}
%%% Using this packages numbers with units are always typed correctly.
\usepackage{siunitx}
%%% To cross out some stuff.
\usepackage{cancel}
%%% To have an enumerate environment with options to get like.
%%% i) bla
%%% ii) lala
%%% typing just \item
\usepackage{enumerate}
%%% To use \bm\mathrm instead of \vec one needs the appropriate greek letters
\usepackage{upgreek}
%%% bold math
\usepackage{bm}
%%% Nice tabular
\usepackage{booktabs}
%%% For iddots (Inverse diagonal triple dot.)
\usepackage{mathdots}
%%% To change the headers. 
\newcommand{\changefont}{%
	\fontsize{9}{11}\selectfont
}
%%% fancy headers
%\usepackage{fancyhdr}
%\pagestyle{fancy}
%\fancyhead[LE]{\changefont\slshape\nouppercase \rightmark} %section
%\fancyhead[RE]{\thepage}
%\fancyhead[RO]{\changefont\slshape\nouppercase \leftmark} % chapter
%\fancyhead[LO]{\thepage}
%\fancyfoot[C]{}

%%% To use mathclap.
\usepackage{mathtools}
%%% To have a set of bigger braces.
%%% I found out the guy who made that was at at the university of Lille, the city I was born! =)
\usepackage{yhmath}
%%% subfigures, subfloat used in week8.
\usepackage{subfig}

%%% mmaetz: This is for the change notes
%\usepackage[deletedmarkup=none]{changes}

%%% Use this option instead of the above one to make the red stuff disappear and the blue stuff become black. (And also remove the footnotes done with the change package.)
%%% Below are the options used for the change package. Note that the default is to cross out the deleted stuff but this doesn't work well in a math environment so the default settings have been changed.
%%% They have been commented out because all change markups have been removed.
%\usepackage[deletedmarkup=none]{changes}
%\setdeletedmarkup{\textcolor{red!75!black}{#1}}
%\setauthormarkupposition{left}
%\setremarkmarkup{\footnote{#1: \textcolor{Changes@Color#1}{#2}}}
%\setremarkmarkup{\marginpar{#1:#2}}
%%% mmaetz: To suppress the change notes put the final option like that:
%\usepackage[final]{changes}
%%% mmaetz: I'm using an authors id
%\definechangesauthor[name={Marc Maetz},color=blue!50!black]{MM}

%%% mmaetz: Can be useful for something but forgot what and I'm not using it here.
%\usepackage{etoolbox}
%%% mmaetz: Just discovered breqn which provides automatic line breaking. Very nice, more powerful but requires a bit of getting used to it.
%%% flexisym is needed by breqn
\usepackage{flexisym}
\usepackage{breqn}
%%% To make the references clickable
%%% The color settings are in the tikzstuff.tex file.
\usepackage{hyperref}

%\hypersetup{colorlinks=false,linkbordercolor=gray}

%%% mmaetz: For a better readability, I made the vectors bold and roman
%%% note that this is bad practice. Consider something like:
%%% If you find a way to make a curly E upright, you might win the nobel price.
\newcommand{\vtr}[1]{\bm{\mathrm#1}}
\renewcommand{\vec}[1]{\bm{\mathrm#1}}
\newcommand{\s}{\ensuremath{\;}}
\newcommand{\lag}{\ensuremath{\cal L}}
\newcommand{\dslash}{\ensuremath{\displaystyle{\not}}}
\newcommand{\Ta}{\ensuremath{\tilde T}}
\newcommand{\tr}{\ensuremath{\text{tr}}}

%%% These commands have been modified because breqn needs to know if a "|" is left or right for the automatic alignment
\newcommand{\ket}[1]{\ensuremath{\left\lvert {#1} \right\rangle}}
\newcommand{\bra}[1]{\ensuremath{\left\langle {#1} \right\rvert}}
%%% mmaetz: Added negative thin space, added vphantom (makes both sides of the braket of the same height in all cases)
\newcommand{\braket}[2]{\ensuremath{\left\langle {#1}\vphantom{#2}   \middle| {#2}\vphantom{#1} \right\rangle}}
\newcommand{\ketbra}[2]{\ensuremath{ {\ket{#1} \!\bra{#2}}}}
\newcommand{\proj}[1]{\ensuremath{ {\ket{#1} \!\bra{#1}}}}
\newcommand{\sumproj}[1]{\ensuremath{ {\sum_{#1}\proj{#1}}}}
\newcommand{\sand}[3]{\ensuremath{\left\langle {#1}\vphantom{#3} \right\rvert{#2} \left\lvert{#3}\vphantom{#1}\right\rangle}}


\newcommand{\vevj}[2]{\left\langle {#1} \right\langle_{#2} }
\newcommand{\vev}[1]{\left\langle {#1} \right\rangle }


\newcommand{\comm}[2]{\left[ {#1}, {#2} \right] }
\newcommand{\acomm}[2]{\left\{ {#1}, {#2} \right\} }

\newcommand{\norm}[1]{\left\lvert {#1} \right\rvert }

%%% mmaetz: For the differential.
\newcommand{\td}{\mathrm d }
%%% vphantom command to get bigger braces where needed
\newcommand{\vaa}{\vphantom{A^{A}}}
\newcommand{\vaaa}{\vphantom{{A^{A}}^{a}}}
\newcommand{\dls}{\delimitershortfall=-1pt}

%%% calculus: (Part of what I found some time ago somewhere in the internet.)
\newcommand{\ddd}[2]{\frac{\mathrm d ^{2} #1}{\mathrm d #2 ^{2}}}
\newcommand{\dd}[2]{\frac{\mathrm d #1}{\mathrm d #2}}
\newcommand{\ddn}[3]{\frac{\mathrm d ^{#1} #2}{\mathrm d #3 ^{#1}}}
\newcommand{\pdd}[2]{\frac{\partial #1}{\partial #2}}
\newcommand{\pddd}[2]{\frac{\partial^{2} #1}{\partial #2 ^{2}}}
\newcommand{\pdddm}[3]{\frac{\partial^{2} #1}{\partial #2 \partial #3}}
\newcommand{\pddn}[3]{\frac{\partial^{#1} #2}{\partial #3 ^{#1}}}



\newtheorem{expl}{Example}[chapter]
\newtheorem{remk}{Remark}[chapter]
\newtheorem{corl}{Corollary}[chapter]
\newtheorem{thrm}{Theorem}[chapter]
\newtheorem{exer}{Exercise}[chapter]

%% Proof environment with a small square as a "qed" symbol
%\theoremstyle{nonumberplain}
%\newtheorem{proof}{Proof}
%\qedsymbol{\qed}
%\theoremsymbol{\qed}

%% Memoir layout setup

%% NOTE: You are strongly advised not to change any of them unless you
%% know what you are doing.  These settings strongly interact in the
%% final look of the document.

% Dependencies
\usepackage{array}
\counterwithout{chapter}{section}

% Define the default sans serif font as the lighter computer modern bright by
% D. Knuth.
\renewcommand{\sfdefault}{cmbr}

%%% Define a nice orange color and use it for hyperref
%%% Needs xcolor
\usepackage{xcolor}
\definecolor{rioday}{RGB}{255,166,0} 
\hypersetup{colorlinks=false,linkbordercolor=rioday}


% Turn extra space before chapter headings off.

% Chapter style redefinition
\makeatletter
\newcommand\thickhrulefill{\leavevmode \leaders%
\hrule height 6.25pt depth -3.25pt \hfill \kern \z@}
\setlength\midchapskip{10pt}
\makechapterstyle{VZ14}{
  \renewcommand\chapternamenum{}
  \renewcommand\printchaptername{}
  \renewcommand\chapnamefont{\Large\scshape}
  \renewcommand\printchapternum{%
    \chapnamefont\null\thickhrulefill\quad
    \@chapapp\space\thechapter\quad\thickhrulefill}
  \renewcommand\printchapternonum{%
    \par\thickhrulefill\par\vskip\midchapskip
    \hrule\vskip\midchapskip
  }
  \renewcommand\chaptitlefont{\Huge\scshape\centering}
  \renewcommand\afterchapternum{%
    \par\nobreak\vskip\midchapskip\hrule\vskip\midchapskip}
  \renewcommand\afterchaptertitle{%
    \par\vskip\midchapskip
\hrule\nobreak\vskip\afterchapskip}
}

% Set the way pages are layed out (headers and page numbering)
\pagestyle{ruled}
%\if@twoside
  %\pagestyle{Ruled}
%\else
  %\pagestyle{ruled}
%\fi

% Use the newly defined style
\chapterstyle{VZ14}

% Redefine sectional headings to contain rules
%\renewcommand{\section}{\@startsection{section}{1}{0mm}%
%{-2\baselineskip}{0.8\baselineskip}%
%{\hrule depth 0.2pt width\textwidth\hrule depth1.5pt%
%width0.25\textwidth\vspace*{1.2em}\Large\bfseries\sffamily}}

%\renewcommand{\subsection}{\@startsection{subsection}{2}{0mm}%
%{-2\baselineskip}{0.8\baselineskip}%
%{\hrule depth 0.2pt width\textwidth\hrule depth1pt width0.25\textwidth\vspace*{0.8em}\large\bfseries\sffamily}}

%\renewcommand{\subsubsection}{\@startsection{subsubsection}{3}{0mm}%
%{-2\baselineskip}{0.8\baselineskip}%
%{\large\bfseries\sffamily}}

\setparaheadstyle{\normalsize\bfseries\sffamily}
\setsubparaheadstyle{\normalsize\bfseries\sffamily}

% Set captions to a more separated style for clearness
\captionnamefont{\sffamily\bfseries\footnotesize}
\captiontitlefont{\sffamily\footnotesize}
\setlength{\intextsep}{16pt}
\setlength{\belowcaptionskip}{1pt}

%%% Make a bit of additional space for footnotes
\addtolength{\skip\footins}{4pt}
\renewcommand{\footnoterule}{%
   \kern -7pt                   % call this kerna
   \hrule height 0.4pt width 0.4\columnwidth
   \kern 6.6pt                  % call this kernb
}

% Set section and TOC numbering depth to subsection
\setsecnumdepth{subsection}
\settocdepth{subsection}

% Turn off american style paragraph indentation and add some space to be
% printed when a new paragraph starts.

\setlength{\parindent}{0pt}
\addtolength{\parskip}{2pt}

\newcommand{\professor}[1]{\def\@professor{#1}}
\renewcommand{\maketitlehookb}%
{\vspace{2em}\centering\Large\@professor\vspace{0.3\textheight}}

%% This provides a frontend to set the lecture date into the header
%% The chapter names are usually shorter than the section names. So the date should be at this place.
%\newcommand{\lecturedate}[1]{\def\@lecdate{#1}}
%\makeevenhead{ruled}{\normalfont\leftmark,}{}{\@lecdate}
%%% Make the header the same width as the text
%\makerunningwidth{ruled}{\textwidth}
%\makeheadrule{ruled}{\textwidth}{\normalrulethickness}
\renewcommand{\footruleskip}{-5pt}
\makeatother

% This defines how theorems should look. Best leave as is.
\theoremstyle{plain}
\theoremseparator{:\quad}
\theoremprework{}
\theoremindent2em
\theoremheaderfont{\sffamily\bfseries}
\theorembodyfont{\normalfont}
%\theoremsymbol{}
%% Minimal margin to print two pages on an A4 paper or viewing it on tablets.
\settypeblocksize{17.7cm}{11.8cm}{*}
\setlrmargins{2cm}{*}{*}
\setulmargins{1.6cm}{*}{*}
\setheadfoot{7pt}{20pt}
\setlength{\beforechapskip}{-1.2cm}
\checkandfixthelayout

%%% New definition of square root:
%%% it renames \sqrt as \oldsqrt
%\let\oldsqrt\sqrt
%%% it defines the new \sqrt in terms of the old one
%\def\sqrt{\mathpalette\DHLhksqrt}
%\def\DHLhksqrt#1#2{%
%	\setbox0=\hbox{$#1\oldsqrt{#2\,}$}\dimen0=\ht0
%	\advance\dimen0-0.2\ht0
%	\setbox2=\hbox{\vrule height\ht0 depth -\dimen0}%
%	{\box0\lower0.4pt\box2}}

%%% Theorems. (Copied from www.mitschriften.ethz.ch template)


\title{\Huge Physically-based simulation in Computer Graphics}


\preauthor{}
\author{\LARGE Kevin Wallimann, Andri Schmidt, Marc Maetz}
\postauthor{\Large\\\vspace{1em}
  Disney Research \\ 
  %ETH Zurich,\\
  %8093 Zurich, Switzerland\\
	\vspace{1em}
  %E-mail: {\em babis@phys.ethz.ch}
}
\professor{}
%\professor{Prof. Dr. Markus Gross, Prof. Dr. Marc Pollefeys}

\date{\vspace{1em}\today}

%\author{Babis Anastasiou\\
%  Institute for Theoretical Physics, \\ 
%  ETH Zurich,\\
%  8093 Zurich, Switzerland\\
%  E-mail: {\em babis@phys.ethz.ch}}

\begin{document} 

\begin{titlingpage}
\maketitle
	%\titleS
%\vfill
%Revision \SVNRev{} --- \SVNDate{}
\end{titlingpage}


%%% Color settings and plot settings. If the pictures look bad look at this file!
%\definecolor{scolor}{RGB}{255,220,181} 
%\definecolor{scolor}{RGB}{255,166,0} 
%\colorlet{scolordark}{scolor!50!black}

\definecolor{rioday}{RGB}{255,166,0} 
\colorlet{riomorning}{rioday!65!black}
\colorlet{rionight}{rioday!50!black}
%\definecolor{scolor2}{RGB}{0,255,166} 
%\definecolor{scolor2}{RGB}{100,100,170} 
%\definecolor{scolor2}{RGB}{0,233,255} 
%\definecolor{scolor2}{RGB}{0,182,255} 
%\definecolor{scolor2}{RGB}{0,255,16} 
%\definecolor{scolor2}{RGB}{0,89,255} 
%\colorlet{scolor2dark}{scolor2!50!black}

\definecolor{helsinkiday}{RGB}{0,89,255} 
%%% Hi from 2013. I think it's fascinating that this code could possibly be used until like 2043 or even further.
\colorlet{helsinkimorning}{helsinkiday!65!black}
\colorlet{helsinkinight}{helsinkiday!50!black}
\hypersetup{colorlinks=false,linkbordercolor=rioday}
%%%This is the global setting of the number of samples used to plot the functions in the graphs. 
\tikzset{
	smasa/.style={
		%%%The number of samples with low number.
		%%% 1000 should be enough for a final edition
		%samples=10
		samples=1000
	},
	bigsa/.style={
		%%%The number of samples with high number.
		%%% 10000 should be enough for a final edition.
		%samples=20
		samples=10000
	},
	%%%The standard filling
	sfill/.style={
		fill=rioday
	},
	zyplane/.style={canvas is zy plane at x=#1,very thin},
	zxplane/.style={canvas is zx plane at y=#1,very thin},
	yxplane/.style={canvas is yx plane at z=#1,very thin}
}




%\begin{abstract}
%The subject of the course an  introduction to  quantum 
%field theory. The following topics are discussed: 
%\begin{itemize}
%\item Theory of classical fields.  
%\item Canonical quantization of free fields.
%\item The  Dirac equation and  quantization of the Dirac field
%\item Field Propagation, interacting fields and perturbation theory. 
%\item Cross-sections  and decay rates.
%\item Introduction to QED and the problem of infinities.
%\item One-loop renormalization of QED. 
%\end{itemize}
%\end{abstract}

\tableofcontents 
%\listofchanges
%\bibliographystyle{JHEP}
%\begin{thebibliography}{10}
%\bibitem{srednicki}
%Modern Quantum Mechanics, Sakurai 
%\bibitem{sterman}
%The Feynman Lectures in Physics, Feynman 
%\bibitem{weinberg}
%The Quantum Theory of Fields, Weinberg 
%\end{thebibliography} 
%\newpage

%%% Have decided to reverse the order of the material wrt the first time I taught this. 
%%% Files renamed according to the new order.

%%%%%%%%%%%%%%%%%%%%%%%%%%%%%%%%%%%%%%%%%%%%%%%%%
%%%%% bibliography
%%%%%%%%%%%%%%%%%%%%%%%%%%%%%%%%%%%%%%%%%%%%%%%%%%

\chapter*{Serie 1}
First lets do a sketch as seen in figure \ref{fig:sp} to omit any misunderstanding of the problem.
\begin{figure}[]
	\begin{center}
		\begin{tikzpicture}[decoration=zigzag, mass/.style={label distance=2mm},scale=1.5]
			\fill[black!30] (-1.5,0) rectangle (1.5,0.5);
			\draw[thick] (-1.5,0) -- (1.5,0);
			\begin{scope}[shift={(2,0)}]
				\coordinate[label=above:$y$,thick] (y) at (0,0.5);
				\coordinate[label=right:$x$,thick] (x) at (0.5,0);
				\draw[-latex,thick] (0,0) -- (y);
				\draw[-latex,thick] (0,0) -- (x);
			\end{scope}
			\coordinate[label={[mass]350:$p_1$}] (m1) at (0,0);
			\coordinate[label={[mass]350:$p_2$}] (m2) at (0,-2.5);
			\draw[decorate,thick] (m1) -- (m2);
			\fill[blue,draw=black] (m1) circle (4pt);
			\fill[red,draw=black] (m2) circle (4pt);
			\path[-latex,draw,thick] (-0.5,0) to[left] node{$-L$} (-0.5,-2);
			\path[-latex,draw,thick] (1,0) to[right] node{$-y$} (1,-2.5);
			\draw[dashed,thick] (-0.5,-2) -- (0.5,-2);
		\end{tikzpicture}
	\end{center}
	\caption{Sketch of the spring problem, $p_2$ being not in rest state.}
	\label{fig:sp}
\end{figure}
We would like to find the constants $c_1, c_2$ of the equation
\begin{align}
	y(t)&=c_1 e^{\alpha t}\cos (\beta t)+c_2 e^{\alpha t}\sin(\beta t) - L -\frac{mg}{k},
	\label{eq:gf}
	\intertext{where the constraints are the rest state initial conditions, which means there is no energy in the system except gravitational energy at $t=0$. This has the same physical meaning as the spring having its rest length $L$ \footnote{if there would be no other force than the one of the feather} and no kinetic energy at $t=0$. This can be expressed as}
	 y(t=0) &= -L,
	\label{eq:c1a}
	\shortintertext{and}
		\dot{y}(t=0)&=0.
	\label{eq:c2a}
	 \intertext{The position of $p_2$ computed with the given solution Eq. \ref{eq:gf} is}
	 y(t=0) &= c_1-L-\frac{mg}{k}.
	\label{eq:c1b}
	\intertext{Comparing Eq. \ref{eq:c1a} and Eq. \ref{eq:c1b} gives us}
	c_1&=\frac{mg}{k}.
	\intertext{To get $c_2$ we need to compute $\dot{y}(t=0)$ from  Eq. \ref{eq:gf}}
	\dot{y}(t=0)&=\!\left( c_1\alpha +c_2 \beta -L -\frac{mg}{k} \right),
	\shortintertext{where $c_1=mg/k$ leads to}
	\dot{y}(t=0)&= \frac{mg}{k}\alpha +c_2\beta -L-\frac{mg}{k}.
	\intertext{Using the constraint in Eq. \ref{eq:c2a} finally gives us}
	c_2&=\frac{mg}{\beta k}(1-\alpha)+\frac{L}{\beta}.
\end{align}
\begin{figure}[]
	\begin{center}
		\includegraphics*[width=\textwidth]{graphics/ErrorCvgDamp0_01-crop.pdf}
	\end{center}
	\caption{<++>}
	\label{fig:<++>}
\end{figure}<++>
\begin{figure}[]
	\begin{center}
		\includegraphics*[width=\textwidth]{graphics/ErrorCvgDamp0-crop.pdf}
	\end{center}
	\caption{}
	\label{fig:}
\end{figure}
\begin{figure}[]
	\begin{center}
		\includegraphics*[width=\textwidth]{graphics/ErrorCvgDamp1-crop.pdf}
	\end{center}
	\caption{}
	\label{fig:}
\end{figure}
\begin{figure}[]
	\begin{center}
		\includegraphics*[width=\textwidth]{graphics/StabilityDamp0_01-crop.pdf}
	\end{center}
	\caption{}
	\label{fig:}
\end{figure}
\begin{figure}[]
	\begin{center}
		\includegraphics*[width=\textwidth]{graphics/StabilityDamp0_1-crop.pdf}
	\end{center}
	\caption{}
	\label{fig:}
\end{figure}
\begin{figure}[]
	\begin{center}
		\includegraphics*[width=\textwidth]{graphics/StabilityDamp0-crop.pdf}
	\end{center}
	\caption{}
	\label{fig:}
\end{figure}
\begin{figure}[]
	\begin{center}
		\includegraphics*[width=\textwidth]{graphics/StabilityDamp1_9-crop.pdf}
	\end{center}
	\caption{}
	\label{fig:}
\end{figure}


\end{document}

% LocalWords:  asimple
